\chapter{Vorschrift zur photometrischen Bestimmung von Hydroxymethylfurfural}

Hydroxymethylfurfural (HMF) entsteht bei der thermischen Zersetzung von Fructose und kann z.B. in Honig vorkommen.\\
Die Probe wird mit p-Toluidin und Barbitursäure versetzt und die gefärbte Lösung photometrisch vermessen.

\section{Chemikalien}
- Hydroxymethylfurfural\\
- p-Toluidinlösung, 100g/L in 2-Propanol\\
- Barbitursäurelösung, 5g/L\\
- Kaliumhexacyanoferrat(II)-Trihydrat\\
- Zinkacetat-Dihydrat\\
- VE-Wasser

\section{Bestimmung}
\subsection{Herstellung der Carrez-Lösung I}
15,0g Kaliumhexacyanoferrat(II)-Trihydrat werden in einem 100mL Messkolben in VE-Wasser gelöst und bis zur Ringmarke aufgefüllt.
\subsection{Herstellung der Carrez-Lösung II}
30,0g Zinkacetat-Dihydrat werden in einem 100mL Messkolben in VE-Wasser gelöst und bis zur Ringmarke aufgefüllt.
\subsection{Herstellung der Kalibrierlösungen}
Mit der Kalibrierung soll der Bereich von 5 bis 300ppm HMF abgedeckt werden. Die Kalibrierlösungen sind entsprechend der Probelösung anzusetzen.
\subsection{Herstellung der Probelösung}
10g Honig wird in einen 50mL Messkolben eingewogen und ohne Erwärmen in 20mL VE-Wasser gelöst. Nacheinander werden je 1mL Carrez I und II zugegeben und nach Durchmischen mit VE-Wasser bis zur Ringmarke aufgefüllt. Die Lösung wird durch einen trockenen Faltenfilter filtriert, wobei die ersten 10mL verworfen werden.
\subsection{Photometrische Messung}
In zwei 10mL Messkolben werden je 2mL Probenlösung pipettiert und je 5mL p-Toluidinlösung (Achtung: giftig, PSA benutzen und im Abzug arbeiten) zugegeben. In den einen Messkolben wird 1mL VE-Wasser pipettiert und in den anderen Messkolben 1mL Barbitursäurelösung und beide Mischungen durchmischt.\\ 
Reaktionszeit: 3 - 4 Minuten\\
Achtung der Farbstoff zerfällt nach Ablauf dieser Zeit wieder.\\
Die Messwellenlänge ist mittels eines Lambdascans im sichtbaren Bereich zu bestimmen.
\subsection{Auswertung}
Der Gehalt an HMF in mg/kg berechnet sich wie folgt:\\
\[HMF[mg/kg]=\frac{ E * 1920 }{ m }\]
Hierbei steht E für die Extinktion, m für die Probenmasse und der Festfaktor 1920 setzt sich zusammen aus dem molaren Extinktionskoeffizienten und der Verdünnung und Umrechnung in die gewünschte Einheit mg/kg.\\
Des Weiteren wird eine Kalibriergerade erstellt und die Proben über diese ausgewertet.
Die Ergebnisse sind zu vergleichen und Rückschlüsse zu ziehen.\\
Diese Vorschrift basiert auf der Anleitung zur ``Photometrischen Bestimmung von Hydroxymethylfurfural'' aus dem Buch ``Lebensmittelanalytik \cite{Lebensmittelanalytik}''.