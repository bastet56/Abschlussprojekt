\chapter{Einleitung} 

Durch l�ngere Lagerung und thermische Belastung kann in Honig Hydroxymethylfurfural (HMF) in gr��erer Konzentration entstehen. HMF steht im Verdacht krebserregend zu sein. Zudem ist ein niedriger Gehalt an HMF ein Indikator f�r die Frische und Naturbelassenheit von Honig.
\newline
Eine M�glichkeit das HMF in Honig zu bestimmen ist die photometrische Methode nach WINKLER. Mittels einer Farbreaktion wird ein rot erscheinender Farbstoff erzeugt, der anschlie�end quantifiziert wird. Zur  Bestimmung des HMF-Gehalts wird zun�chst die Methode, wie in dem Buch Lebensmittelanalytik beschrieben, nachgestellt. Da in der zur Verf�gung stehenden Methode lediglich ein Festfaktor zur Quantifizierung angegeben ist wird zus�tzlich eine Kalibrierreihe mit mehreren Punkten durchgef�hrt. Des weiteren wird die Methode auf Reproduzierbarkeit �berpr�ft. Zur Ermittlung des Massenanteils an HMF sollen mehrere Arten der Quantifizierung verwendet werden. Diese sind zum einen die Berechnung �ber den angegebenen Festfaktor, die Verwendung der erstellten Kalibriergerade sowie �ber Aufstockung einer Probe mit anschlie�ender Standardaddition. Es werden acht verschiedene Honige, ein R�bensirup und eine Invertzuckermischung, die bei Raumtemperatur aufbewahrt wurden, vermessen. Au�erdem werden sechs Honige f�r zehn Tage bei 60�C gelagert um einen Anstieg des HMF-Gehalts nachzuweisen. Die Handhabung der Proben ist durch ihre Viskosit�t erschwert. Da der Honig viele verschiedene, zum Teil unl�sliche Bestandteile enth�lt, wird diese Matrix vor der Bestimmung entfernt. Die verwendeten Chemikalien erfordern auf Grund ihrer gef�hrlichen Eigenschaften eine besondere Sorgfalt bei der Handhabung. Ihre Haltbarkeit ist auch bei korrekter Lagerung sehr begrenzt. Der bei der Durchf�hrung entstandene Farbstoff ist nicht stabil und zerf�llt wenige Minuten nach der Aufarbeitung. Deshalb muss jede L�sung zeitlich exakt vermessen werden.