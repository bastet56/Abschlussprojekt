\chapter{Einleitung} 

Durch l�ngere Lagerung und thermische Belastung kann in Honig Hydroxymethylfurfural (HMF) in gr��erer Konzentration entstehen. HMF steht im Verdacht krebserregend zu sein. Zudem ist ein niedriger Gehalt an HMF ein Indikator f�r die Frische und Naturbelassenheit von Honig.
\newline
Eine M�glichkeit das HMF in Honig zu bestimmen ist die photometrische Methode nach WINKLER.