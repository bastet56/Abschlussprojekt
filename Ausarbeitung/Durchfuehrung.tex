\chapter{Durchführung}

Für die Kalibrierung und die Probenvermessung müssen verschiedene Reaktionslösungen angesetzt werden, die einerseits die Matrixeinflüsse verhindern und andererseits für die Farbreaktion benötigt werden. Im folgenden Kapitel sind die verwendeten Chemikalien und Geräte aufgelistet, sowie der genaue Arbeitsablauf beschrieben.

\section{Verwendete Chemikalien}


Um störende Matrixbestandteile der Honigproben vorab zu entfernen, werden die Carrez-Lösungen I und II benötigt. Die für die Farbreaktion verwendeten Reaktionslösungen können im Chemikalienhandel erworben werden. Dies gilt ebenfalls für die Reinsubstanz HMF, die für die Kalibrierlösungen und die Aufstockung benutzt wird. In der nachfolgenden Tabelle \ref{tab:Chemikalienliste} sind alle benötigten Chemikalien aufgelistet.

\begin{table}[htbp]
	\centering
		\begin{tabular}{p{0.18\linewidth}|p{0.13\linewidth}|p{0.1\linewidth}|c|c|c} 
			Chemikalie & CAS/ Artikel-Nr. & Gefahren-symbol & Reinheit & Hersteller & Lot-Nr.\\
			\hline
			Hydroxymethyl-furfural & 67-47-0 & & 97\% & Alfa Aesar & 10189124\\
			\hline
			p-Toluidin-lösung & 18686.2700 & & 100g/L & Bernd Kraft & 1632697\\
			\hline
			Barbitursäure-lösung & 18685.2700 & & 5g/L & Bernd Kraft & 1632696\\
			\hline
			Kaliumhexa-cyanoferrat-(II)-Trihydrat & 14459-95-1 & & $\geq99\%$ & Sigma-Aldrich & SZBC2230V\\
			\hline
			Zinkacetat-Dihydrat & 5970-45-6 & & $\geq99,5\%$ & Merck & A0180402 142\\
			\hline
			VE-Wasser & & & & &
		\end{tabular}
	\caption{Chemikalienliste}
	\label{tab:Chemikalienliste}
\end{table}
%Tabelle einfügen
Die Sicherheitsdatenblätter sind in Anhang xyz zu finden.

\section{Verwendete Geräte}

Für die Probenvorbereitung und um die verschiedenen Lösungen herzustellen, werden verschiedene Laborgeräte verwendet. Diese sind in der folgenden Tabelle \ref{tab:Geräteliste} zusammengefasst.

\begin{table}[htbp]
	\centering
		\begin{tabular}{l|p{0.25\linewidth}|c|c|c} 
			Anzahl & Gerät & Volumen in mL & Genauigkeit & Auslaufzeit\\
			\hline
			16 & Messkolben & 10 & A (+/- 0,040mL) & \\
			\hline
			8 & Messkolben & 50 & A (+/- 0,060mL) & \\
			\hline
			4 & Messkolben & 100 & A (+/- 0,100mL) & \\
			\hline
			4 & Vollpipetten & 1 & AS (+/- 0,006mL) & EX\\
			\hline
			2 & Vollpipetten & 2 & AS (+/- 0,010mL) & EX + 15s\\
			\hline
			1 & Vollpipetten & 5 & AS (+/- 0,015mL) & EX + 15s\\
			\hline
			1 & Vollpipetten & 10 & AS (+/- 0,02mL) & EX + 15s\\
			\hline
			1 & Vollpipetten & 20 & AS (+/- 0,03mL) & EX + 15s\\
			\hline
			1 & Messzylinder & 25 & & \\
			\hline
			diverse & Bechergläser & & & \\
			\hline
			1 & Wägeschiffchen & & & \\
			\hline
			1 & Analysenwaage Sartorius M-pact AX224 & max. 120g & d=0,1mg & \\
			\hline
			1 & UV/VIS-Spektralphotometer Varian Cary® 50 & & & \\
			\hline
			1 & Präzisions-Küvette aus opt. Spezialglas & d=10mm & & \\
		\end{tabular}
	\caption{Geräteliste}
	\label{tab:Geräteliste}
\end{table}

%Tabelle einfügen

\section{Proben}

Es sollen acht verschiedenen Honige, ein Zuckerrübensirup und eine Invertzuckermischung vermessen werden. Die folgende Tabelle \ref{tab:Probenliste} zeigt die Probendetails.

\begin{table}[htbp]
	\centering
		\begin{tabular}{p{0.1\linewidth}|p{0.18\linewidth}|c|c|p{0.2\linewidth}|c} 
			Proben-nummer & Probe & Hersteller & Ablaufdatum & Herkunft & Lot-Nr.\\
			\hline
			1 & Flotte Biene Frühlings-blütenhonig & Langnese &  & EG- und Nicht-EG-Länder & \\
			\hline
			2 & Flotte Biene Gebirgs-blütenhonig & Langnese &  & EG- und Nicht-EG-Länder & \\
			\hline
			3 & Sommer-blütenhonig & Vom Land &  & EG- und Nicht-EG-Länder & \\
			\hline
			4 & Blütenhonig & Goldland &  & EG- und Nicht-EG-Länder & \\
			\hline
			5 & Mexico & Biophar &  &  & \\
			\hline
			6 & Ägäis & Breitsamer &  &  & \\
			\hline
			7 & Waldhonig & Breitsamer &  &  & \\
			\hline
			8 & Zuckerrüben-sirup & Grafschafter &  &  & \\
			\hline
			9 & Winterfutter & Imker B. Hahl &  & Walldorf, Deutschland & \\
			\hline
			10 & Invertzucker &  &  &  & \\
		\end{tabular}
	\caption{Probenliste}
	\label{tab:Probenliste}
\end{table}

%Tabelle einfüge

\section{Ansetzen der Reaktionslösungen}

Für die Probenaufbereitung werden zwei Carrez-Lösungen benötigt.\\ 
Für die Carrez-Lösung I wird am ersten Praktikumstag 15,1805g Kaliumhexacyanoferrat in einen 100mL Messkolben eingewogen, mit Wasser gelöst und bis zur Ringmarke aufgefüllt.\\ 
Die Carrez-Lösung II wurde zweimal angesetzt, da nach einer Woche Lagerung feste Partikel im Messkolben festgestellt wurden. Am ersten Praktikumstag wurde für die Carrez-Lösung II 30,1485g Zinkacetat in einen 100mL Messkolben eingewogen, mit Wasser im Ultraschallbad gelöst und bis zur Ringmarke aufgefüllt. Da sich das Zinkacetat schlecht auflöste, wurde die Lösung beim zweiten Ansetzen am dritten Praktikumstag leicht erwärmt. Für die zweite Lösung wurde 30,0504g Zinkacetat eingewogen.\\ 
Die p-Toluidinlösung und die Barbitursäurelösung mussten nicht angesetzt werden, da sie in der benötigten Konzentration zur Verfügung standen. In der p-Toluidinlösung waren eine Woche nach Anbruch Feststoffpartikel enthalten.

\section{Ansetzen der Stammlösungen}

Für die Kalibrierreihe werden zwei Stammlösungen mit unterschiedlicher HMF-Konzentration angesetzt. Die Berechnung der benötigten Einwaagen ist in Kapitel \ref{chap:Planung} nachzulesen.\\
Die HMF-Reinsubstanz wird auf der Analysenwaage in einem Wägeschiffchen eingewogen, mit VE-Wasser in einen 10mL Messkolben überführt und bis zur Ringmarke aufgefüllt.\\

Einwaage SL1: 54,8 mg\\
Einwaage SL2: 151,6 mg\\

Aus den Einwaage wird über Formel xyz die HMF-Konzentration der beiden Stammlösungen berechnet. Die Konzentration der Stammlösung 1 beträgt 5480 mg/L und die Konzentration der Stammlösung 2 beträgt 15160 mg/L.\\ 
Von beiden Stammlösungen wird je ein Milliliter in jeweils einen 100mL Messkolben überführt und mit VE-Wasser bis zur Ringmarke aufgefüllt. Somit ergibt sich für die Stammlösung 1.1 eine HMF-Konzentration von 54,8 mg/L und für die Stammlösung 2.1 eine HMF-Konzentration von 151,6 mg/L. 

\section{Herstellung und Vermessung der Kalibrierlösungen}

Für die Kalibrierlösungen werden aliquote Teile der beiden Stammlösungen 1.1 und 2.1 mit verschiedenen Vollpipetten in 50mL Messkolben abgefüllt und mit etwas VE-Wasser vermischt. Anschließend werden je 2mL Carrez-Lösung I und II mit eine Vollpipette hinzugefügt. Nach Durchmischen der Lösungen wird mit VE-Wasser bis zur Ringmarke aufgefüllt. Dabei fallen störende Matrixbestandteile als unlösliche Partikel aus. Die Lösungen werden über einen Faltenfilter filtriert, wobei die ersten 10mL Filtrat verworfen werden. Von dem restlichen Filtrat werden 2mL mit einer Vollpipette entnommen und in einen 10mL Messkolben überführt. In die Kolben werden außerdem jeweils 5mL p-Toluidinlösung und 1mL Barbitursäurelösung pipettiert. Von dem Filtrat der ersten Kalibrierlösung wird ebenfalls die Lösung für den Blindwert angesetzt. Hierbei werden 5mL p-Toluidinlösung zugegeben, aber anstelle der Barbitursäurelösung 1mL VE-Wasser zugesetzt. Zum Homogenisieren werden die Messkolben verschlossen und mehrmals invertiert. Da es sich bei der Farbreaktion um eine Zeitreaktion handelt, werden die Kalibrierlösungen und der Blindwert vor der Vermessung vier Minuten stehen gelassen. Da der Farbkomplex nach dieser Zeit wieder zerfällt, müssen die Lösungen zügig vermessen werden.

%Hinweise auf Beispielrechnung unter Planung?

Mit der sechsten Kalibrierlösung wird die Wellenlänge des Absorptionsmaximums zwischen 200 und 800nm bestimmt. Der Ausdruck der Wellenlängenbestimmung ist in Anhang X zu finden.\\

\[
  \lambda_{max} = 550 nm
\]

Dies entspricht der in der Literatur angegebenen Wellenlänge zur Vermessung der Farbkomplexes.\\
Die Kalibrierlösungen werden bei 550nm gegen den Blindwert vermessen und eine Kalibriergerade erstellt. Um Messfehler auszuschließen wird jede Kalibrierlösung dreimal gemessen.\\
In der nachfolgenden Tabelle \ref{tab:Kalibrierlösungen} sind die Kalibrierlösungen aufgelistet.

\begin{table}[htbp]
	\centering
		\begin{tabular}{p{0.1\linewidth}|p{0.1\linewidth}|p{0.15\linewidth}|p{0.15\linewidth}|p{0.15\linewidth}|c} 
			Kalibrier-lösung & Stamm-lösung & Volumen Stamm-lösung\newline in mL & Massenkon-zentration berechnet\newline in mg/L & Massenanteil berechnet\newline in mg/kg & Extinktion\\
			\hline
			1 & 1.1 & 1 & 1,096 & 5,480 & 0,0400\\
			\hline
			2 & 1.2 & 1 & 3,032 & 15,16 & 0,1353\\
			\hline
			3 & 1.1 & 5 & 5,480 & 27,40 & 0,1687\\
			\hline
			4 & 1.1 & 10 & 10,96 & 54,80 & 0,3214\\
			\hline
			5 & 1.2 & 10 & 30,32 & 151,6 & 0,9453\\
			\hline
			6 & 1.2 & 20 & 60,64 & 303,2 & 1,8987\\
		\end{tabular}
	\caption{Kalibrierlösungen}
	\label{tab:Kalibrierlösungen}
\end{table}

%Tabelle Kalibrierlösungen 

\section{Herstellung und Vermessung der Probelösungen}

Für die Probelösungen werden jeweils ca. 10g Probe in einen 50mL Messkolben eingewogen und in 20mL VE-Wasser gelöst. Die folgende Tabelle \ref{tab:Probeneinwaage} enthält die Lagertemperatur und die Einwaage der einzelnen Proben. 

\begin{table}[htbp]
	\centering
		\begin{tabular}{c|c|c} 
			Probennummer & Lagertemperatur in $^\circ$C & Einwaage in g\\
			\hline
			1 & 25 & 10,2926\\
			\hline
			1 & 60 & 11,0590\\
			\hline
			2 & 25 & 10,2548\\
			\hline
			2 & 60 & 10,0891\\
			\hline
			3 & 25 & 11,3998\\
			\hline
			3 & 60 & 10,2277\\
			\hline
			4 & 25 & 9,9125\\
			\hline
			4 & 60 & 10,0411\\
			\hline
			5 & 25 & 10,0203\\
			\hline
			5 & 60 & 10,2194\\
			\hline
			6 & 25 & 10,0776\\
			\hline
			6 & 60 & 10,0968\\
			\hline
			7 & 25 & 10,4448\\
			\hline
			8 & 25 & 10,4160\\
			\hline
			9 & 25 & 10,1488\\
			\hline
			10 & 25 & 9,9153\\
		\end{tabular}
	\caption{Probeneinwaage}
	\label{tab:Probeneinwaage}
\end{table}
%Tabelle mit Einwaagen

Jeder Probelösung werden je 2mL Carrez-Lösung I und II zugesetzt und die Messkolben nach dem Homogenisieren mit VE-Wasser bis zur Ringmarke aufgefüllt. Die ausgefallenen Partikel werden über Faltenfilter abfiltriert. Dabei werden die ersten 10mL des Filtrats verworfen. Von dem Filtrat werden mit einer Vollpipette jeweils zweimal 2mL entnommen und in zwei 10mL Messkolben überführt. In einem Messkolben wird der Blindwert angesetzt, im anderen die zu vermessende Probe. In beide Messkolben werden 5mL p-Toluidinlösung hinzugefügt. Mit einer 1mL Vollpipette wird dem Blindwert 1mL VE-Wasser zugegeben und der Probelösung 1mL Barbitursäurelösung. Beide Messkolben bleiben nach dem Homogenisieren für 4 Minuten stehen. Die Messung der Probe erfolgt danach bei 550nm gegen den jeweiligen Blindwert. Die Proben werden jeweils dreimal vermessen um Messfehler auszuschließen.

\section{Mehrfachbestimmung und Aufstockung einer Probe}

Für eine Mehrfachbestimmung wird die Honigprobe 3 am zweiten Praktikumstag einmal und am dritten Praktikumstag sechsmal eingewogen, wie oben beschrieben mit Reaktionslösungen versetzt und anschließend vermessen. Außerdem werden zwei zusätzliche Einwaagen der Honigprobe 3 einmal mit 10mL der Stammllösungen 1.1 und einmal mit 10mL der Stammlösung 2.1 versetzt und so der HMF-Gehalt aufgestockt. Diese beiden Proben werden ebenfalls wie oben beschrieben behandelt und vermessen.
