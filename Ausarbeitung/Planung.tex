\chapter{Planung}

\label{chap:Planung}

\section{Honig}

\section{Bildung von HMF}

\section{Nachweis nach Winkler}
%Gefahr durch p-Touluidinlösung

\section{alternative Nachweismethoden}

Neben dem HMF-Nachweis nach Winkler gibt es noch weitere Methoden um HMF in zuckerhaltigen Produkten nachzuweisen.\\
Es gibt zum Beispiel die Möglichkeit das HMF mit Diethylether aus der Probe zu extrahieren und mit einer Resorcinlösung, auch Seliwanoff-Reagenz genannt, als roten Farbstoff sichtbar zu machen. Da das Resorcin auch mit der vorhandenen Fructose reagiert, ist für einen korrekten Nachweis eine saubere Extraktion unbedingt notwendig. Allerdings sind sowohl das Resorcin als auch der Dieethylether als gesundheitsschädlich eingestuft und der Versuch muss im Abzug durchgeführt werden. %Quelle FWG Singen 
Hierbei handelt es sich um den Fieheschen Nachweis. %Quelle Springer\\
Alternativ kann HMF auch per Gaschromatographie oder HPLC bestimmt werden. Hierfür müssen die Proben in einer säulengängigen Form vorliegen und die Methoden für eine Quantifizierung mit einem HMF-Standard kalibriert werden. %Quelle Patent
Alle hier genannten HMF-Nachweise sind mit hohem chemischem und apparativem Aufwand verbunden. Außerdem sind die verwendeten Chemikalien gesundheitsschädlich oder sogar giftig und müssen deshalb mit größter Vorsicht gehandhabt werden. Aus diesen Gründen wurde von Merck ein einfacher Schnelltest entwickelt. Dieser erfolgt mit Teststäbchen, die mit zwei Reaktinslösungen belegt sind. Die Stäbchen müssen nur in die Probe eingetaucht und dann in einem Reflektometer vermessen werden. Hierbei erfolgt die Farbreaktion auf dem Teststäbchen. Das Gerät kann HMF-Konzentrationen zwischen 1 und 60 mg/L erfassen. Höher konzentrierte Proben müssen verdünnt und das Messergebnis mit dieser Verdünnung verrechnet werden. Die Farbreaktion ist zeitabhängig, deshalb muss auch hierbei die Reaktionszeit genau eingehalten werden. %Quelle Merck Packungsbeilage

\section{Funktionsweise eines UV/VIS-Spektroskops}

\section{Funktion Carrez-Lösung}

\section{Organisation des Abschlussprojektes}
Während den Recherchen zur Durchführung unseres Abschlussprojektes entdeckten wir, dass die beiden Reaktionslösungen (p-Toluidinlösung und Barbitursäure) in der benötigten Konzentration im Chemikalienhandel erhältlich sind. Über die Chemikalienbeschaffung der BASF konnten sowohl der benötigte HMF-Standard als auch die beiden Reaktionslösungen bestellt werden. Somit entfällt das Ansetzen der beiden Lösungen während des Praktikums. Es konnte von Seiten des Herstellers keine Aussage über die Haltbarkeit der p-Toluidinlösung nach Öffnung der Flasche getroffen werden. Die selbst angesetzte Lösung wäre nur drei Tage haltbar gewesen. Die restlichen Chemikalien stellte die Berufsschule zur Verfügung. \\
Zehn Tage vor dem zweiten Praktikumstag wurden die zugekauften Chemikalien im Chemikalienkühlschrank der Biologie in der BBSN Ludwigshafen eingelagert. Am gleichen Tag wurden auch die sechs Honigproben für den Lagertest bei 60°C abgefüllt und in einem Wärmeschrank der Biologie abgestellt.
Für die Durchführung des Abschlusspraktikums wurden anderthalb Praktikumstage angesetzt. Auf Grund von Problemen während des Praktikums wurde ein zusätzlicher Tag benötigt. \\
Für die vier Stunden des ersten Praktikumstages wurde das Ansetzen der beiden Carrez-Lösungen, eine erste Probemessung mit einer Honigprobe ohne und mit Temperaturlagerung und das Erstellen der Kalibriergeraden vorgesehen.\\
Am zweiten Praktikumstag sollten die Proben vermessen werden. Da hierbei erkannt wurde, dass der Farbstoff nach einiger Zeit zerfällt, wurde ein weiterer Praktikumstag eingeplant.\\
Während dem dritten Tag wurde die Kalibrierung wiederholt und eine Sechsfachanalyse, sowie eine Aufstockung mit HMF einer Honigprobe durchgeführt.
 