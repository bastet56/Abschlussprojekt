\chapter{Planung}

\label{chap:Planung}

\section{Honig}

\section{Bildung von HMF}

\section{Nachweis nach Winkler}
%Gefahr durch p-Touluidinlösung

\section{alternative Nachweismethoden}

\section{Funktionsweise eines UV/VIS-Spektroskops}

\section{Funktion Carrez-Lösung}

\section{Organisation des Abschlussprojektes}
Während den Recherchen zur Durchführung unseres Abschlussprojektes entdeckten wir, dass die beiden Reaktionslösungen (p-Toluidinlösung und Barbitursäure) in der benötigten Konzentration im Chemikalienhandel erhältlich sind. Über die Chemikalienbeschaffung der BASF konnten sowohl der benötigte HMF-Standard als auch die beiden Reaktionslösungen bestellt werden. Somit entfällt das Ansetzen der beiden Lösungen während des Praktikums. Es konnte von Seiten des Herstellers keine Aussage über die Haltbarkeit der p-Toluidinlösung nach Öffnung der Flasche getroffen werden. Die selbst angesetzte Lösung wäre nur drei Tage haltbar gewesen. Die restlichen Chemikalien stellte die Berufsschule zur Verfügung. \\
Zehn Tage vor dem zweiten Praktikumstag wurden die zugekauften Chemikalien im Chemikalienkühlschrank der Biologie in der BBSN Ludwigshafen eingelagert. Am gleichen Tag wurden auch die sechs Honigproben für den Lagertest bei 60°C abgefüllt und in einem Wärmeschrank der Biologie abgestellt.
Für die Durchführung des Abschlusspraktikums wurden anderthalb Praktikumstage angesetzt. Auf Grund von Problemen während des Praktikums wurde ein zusätzlicher Tag benötigt. \\
Für die vier Stunden des ersten Praktikumstages wurde das Ansetzen der beiden Carrez-Lösungen, eine erste Probemessung mit einer Honigprobe ohne und mit Temperaturlagerung und das Erstellen der Kalibriergeraden vorgesehen.\\
Am zweiten Praktikumstag sollten die Proben vermessen werden. Da hierbei erkannt wurde, dass der Farbstoff nach einiger Zeit zerfällt, wurde ein weiterer Praktikumstag eingeplant.\\
Während dem dritten Tag wurde die Kalibrierung wiederholt und eine Sechsfachanalyse, sowie eine Aufstockung mit HMF einer Honigprobe durchgeführt.
 