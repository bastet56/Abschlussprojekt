%% Basierend auf einer TeXnicCenter-Vorlage von Mark Müller
%%%%%%%%%%%%%%%%%%%%%%%%%%%%%%%%%%%%%%%%%%%%%%%%%%%%%%%%%%%%%%%%%%%%%%%

% Wählen Sie die Optionen aus, indem Sie % vor der Option entfernen  
% Dokumentation des KOMA-Script-Packets: scrguide

%%%%%%%%%%%%%%%%%%%%%%%%%%%%%%%%%%%%%%%%%%%%%%%%%%%%%%%%%%%%%%%%%%%%%%%
%% Optionen zum Layout des Artikels                                  %%
%%%%%%%%%%%%%%%%%%%%%%%%%%%%%%%%%%%%%%%%%%%%%%%%%%%%%%%%%%%%%%%%%%%%%%%
\documentclass[%
%5paper,                            % alle weiteren Papierformat einstellbar
%landscape,                     % Querformat
11pt,                               % Schriftgröße (12pt, 11pt (Standard))
%BCOR1cm,                           % Bindekorrektur, bspw. 1 cm
%DIVcalc,                           % führt die Satzspiegelberechnung neu aus
%                                             s. scrguide 2.4
%twoside,                           % Doppelseiten
%twocolumn,                     % zweispaltiger Satz
%halfparskip*,              % Absatzformatierung s. scrguide 3.1
%headsepline,                   % Trennline zum Seitenkopf  
%footsepline,                   % Trennline zum Seitenfuß
%titlepage,                     % Titelei auf eigener Seite
%normalheadings,            % Überschriften etwas kleiner (smallheadings)
%idxtotoc,                      % Index im Inhaltsverzeichnis
liststotoc,                 % Abb.- und Tab.verzeichnis im Inhalt
bibtotoc,                       % Literaturverzeichnis im Inhalt
%abstracton,                    % Überschrift über der Zusammenfassung an   
%leqno,                         % Nummerierung von Gleichungen links
%fleqn,                             % Ausgabe von Gleichungen linksbündig
%draft                              % überlangen Zeilen in Ausgabe gekennzeichnet
]{scrreprt}

\usepackage{geometry}
\geometry{a4paper, top=20mm, left=28mm, right=30mm, bottom=30mm, headsep=10mm, footskip=15mm}

\usepackage{setspace}
\onehalfspacing

%\pagestyle{empty}      % keine Kopf und Fußzeile (k. Seitenzahl)
%\pagestyle{headings}   % lebender Kolumnentitel  


%% Deutsche Anpassungen %%%%%%%%%%%%%%%%%%%%%%%%%%%%%%%%%%%%%
\usepackage[ngerman]{babel}
\usepackage[utf8]{inputenc}
\usepackage[T1]{fontenc}

\usepackage{lmodern} %Type1-Schriftart für nicht-englische Texte

\usepackage{hyperref}

\usepackage{bibgerm}

\usepackage{array}
\newcolumntype{L}[1]{>{\raggedright\let\newline\\\arraybackslash\hspace{0pt}}m{#1}}
\newcolumntype{C}[1]{>{\centering\let\newline\\\arraybackslash\hspace{0pt}}m{#1}}
\newcolumntype{R}[1]{>{\raggedleft\let\newline\\\arraybackslash\hspace{0pt}}m{#1}}

\usepackage[font=small,labelfont=bf]{caption}
\captionsetup{labelsep=space,justification=justified,singlelinecheck=off}

\usepackage{lastpage}
\usepackage{fancyhdr}

%% Packages für Grafiken & Abbildungen %%%%%%%%%%%%%%%%%%%%%%
\usepackage{graphicx} %%Zum Laden von Grafiken
\usepackage{subfigure} %%Teilabbildungen in einer Abbildung
%\usepackage{pst-all} %%PSTricks - nicht verwendbar mit pdfLaTeX

%% Beachten Sie:
%% Die Einbindung einer Grafik erfolgt mit \includegraphics{Dateiname}
%% bzw. über den Dialog im Einfügen-Menü.
%% 
%% Im Modus "LaTeX => PDF" können Sie u.a. folgende Grafikformate verwenden:
%%   .jpg  .png  .pdf  .mps
%% 
%% In den Modi "LaTeX => DVI", "LaTeX => PS" und "LaTeX => PS => PDF"
%% können Sie u.a. folgende Grafikformate verwenden:
%%   .eps  .ps  .bmp  .pict  .pntg


%% Bibliographiestil %%%%%%%%%%%%%%%%%%%%%%%%%%%%%%%%%%%%%%%%%%%%%%%%%%
%\usepackage{natbib}

\begin{document}

\pagestyle{empty} %%Keine Kopf-/Fusszeilen auf den ersten Seiten.
\setlength{\parindent}{0cm}
\pagenumbering{arabic}

\begin{titlepage}
  \begin{center}
    \vspace*{2cm}
      \Large{Berufsbildende Schule Naturwissenschaften}\\
      \smallskip
      \Large{Ludwigshafen}\\
      \smallskip
      \Large{Fachschule}\\
      \smallskip
      \Large{Fachbereich Chemietechnik}\\
      \smallskip
      \Large{Schwerpunkt Labortechnik}\\
      \smallskip
      \Large{Schuljahr 2014/2015}\\
      \smallskip
      \Large{Projektarbeit im Rahmen des Lernmoduls}
      \smallskip
      
    \vspace{3cm}

    \vspace{0.5\baselineskip} {\Huge \textbf{Abschlussprojekt}\\\vspace{1cm} \huge Die Bestimmung des Massenanteils von Hydroxymethylfurfural (HMF) mittels UV/VIS Spektroskopie in Honig unter Berücksichtigung von Lagerdauer und thermischem Einfluss}
  \end{center}

  \vfill

  {\large
    \begin{tabular}[l]{ll}
      Vorgelegt von: & Sabine Klein und Christian Rasch\\
      Betreuer: & Dr. Gerhard Peiter\\
      Datum der Abgabe: & 02.06.2015
    \end{tabular}
  }

\end{titlepage}

\thispagestyle{empty}

\vspace*{100pt} Ich versichere, dass ich diese Masterarbeit selbstst\"andig verfasst und nur die
angegebenen Quellen und Hilfsmittel verwendet habe.

\vspace*{50pt}


Abgabedatum: 29.08.2012 \newpage

%%%%%%%%%%%%%%%% END OF TITLEPAGE %%%%%%%%%%%%%%%%%%%%%%%%%%

%% Widmungsseite %%%%%%%%%%%%%%%%%%%%%%%%%%%%%%%%%%%%%%%%%%%%%%%%%%%%%%
%\dedication{Widmung}

%% Zusammenfassung nach Titel, vor Inhaltsverzeichnis %%%%%%%%%%%%%%%%%

\newpage
\begin{abstract}
1000 Zeichen Text in Deutsch und Englisch. 
\end{abstract}
\newpage
\begingroup
\renewcommand*{\chapterpagestyle}{empty}
\pagestyle{empty}
\tableofcontents            % Inhaltsverzeichnis
\endgroup

\newpage
\pagestyle{fancy}
\fancyhead{}
\renewcommand{\headrulewidth}{0pt}
\fancyfoot{}
\fancypagestyle{plain}{%
\fancyfoot[ER,OR]{\thepage\ of \pageref*{LastPage}}}

\rfoot{\thepage\ of \pageref*{LastPage}}
\chapter{Einleitung} 

Durch l�ngere Lagerung und thermische Belastung kann in Honig Hydroxymethylfurfural (HMF) in gr��erer Konzentration entstehen. HMF steht im Verdacht krebserregend zu sein. Zudem ist ein niedriger Gehalt an HMF ein Indikator f�r die Frische und Naturbelassenheit von Honig.
\newline
Eine M�glichkeit das HMF in Honig zu bestimmen ist die photometrische Methode nach WINKLER.
\chapter{Planung}

\label{chap:Planung}

\section{Honig}

\section{Bildung von HMF}

\section{Nachweis nach Winkler}
%Gefahr durch p-Touluidinlösung

\section{alternative Nachweismethoden}

Neben dem HMF-Nachweis nach Winkler gibt es noch weitere Methoden um HMF in zuckerhaltigen Produkten nachzuweisen.\\
Es gibt zum Beispiel die Möglichkeit das HMF mit Diethylether aus der Probe zu extrahieren und mit einer Resorcinlösung, auch Seliwanoff-Reagenz genannt, als roten Farbstoff sichtbar zu machen. Da das Resorcin auch mit der vorhandenen Fructose reagiert, ist für einen korrekten Nachweis eine saubere Extraktion unbedingt notwendig. Allerdings sind sowohl das Resorcin als auch der Dieethylether als gesundheitsschädlich eingestuft und der Versuch muss im Abzug durchgeführt werden. %Quelle FWG Singen 
Hierbei handelt es sich um den Fieheschen Nachweis. %Quelle Springer\\
Alternativ kann HMF auch per Gaschromatographie oder HPLC bestimmt werden. Hierfür müssen die Proben in einer säulengängigen Form vorliegen und die Methoden für eine Quantifizierung mit einem HMF-Standard kalibriert werden. %Quelle Patent
Alle hier genannten HMF-Nachweise sind mit hohem chemischem und apparativem Aufwand verbunden. Außerdem sind die verwendeten Chemikalien gesundheitsschädlich oder sogar giftig und müssen deshalb mit größter Vorsicht gehandhabt werden. Aus diesen Gründen wurde von Merck ein einfacher Schnelltest entwickelt. Dieser erfolgt mit Teststäbchen, die mit zwei Reaktinslösungen belegt sind. Die Stäbchen müssen nur in die Probe eingetaucht und dann in einem Reflektometer vermessen werden. Hierbei erfolgt die Farbreaktion auf dem Teststäbchen. Das Gerät kann HMF-Konzentrationen zwischen 1 und 60 mg/L erfassen. Höher konzentrierte Proben müssen verdünnt und das Messergebnis mit dieser Verdünnung verrechnet werden. Die Farbreaktion ist zeitabhängig, deshalb muss auch hierbei die Reaktionszeit genau eingehalten werden. %Quelle Merck Packungsbeilage

\section{Funktionsweise eines UV/VIS-Spektroskops}

\section{Funktion Carrez-Lösung}

\section{Organisation des Abschlussprojektes}
Während den Recherchen zur Durchführung unseres Abschlussprojektes entdeckten wir, dass die beiden Reaktionslösungen (p-Toluidinlösung und Barbitursäure) in der benötigten Konzentration im Chemikalienhandel erhältlich sind. Über die Chemikalienbeschaffung der BASF konnten sowohl der benötigte HMF-Standard als auch die beiden Reaktionslösungen bestellt werden. Somit entfällt das Ansetzen der beiden Lösungen während des Praktikums. Es konnte von Seiten des Herstellers keine Aussage über die Haltbarkeit der p-Toluidinlösung nach Öffnung der Flasche getroffen werden. Die selbst angesetzte Lösung wäre nur drei Tage haltbar gewesen. Die restlichen Chemikalien stellte die Berufsschule zur Verfügung. \\
Zehn Tage vor dem zweiten Praktikumstag wurden die zugekauften Chemikalien im Chemikalienkühlschrank der Biologie in der BBSN Ludwigshafen eingelagert. Am gleichen Tag wurden auch die sechs Honigproben für den Lagertest bei 60°C abgefüllt und in einem Wärmeschrank der Biologie abgestellt.
Für die Durchführung des Abschlusspraktikums wurden anderthalb Praktikumstage angesetzt. Auf Grund von Problemen während des Praktikums wurde ein zusätzlicher Tag benötigt. \\
Für die vier Stunden des ersten Praktikumstages wurde das Ansetzen der beiden Carrez-Lösungen, eine erste Probemessung mit einer Honigprobe ohne und mit Temperaturlagerung und das Erstellen der Kalibriergeraden vorgesehen.\\
Am zweiten Praktikumstag sollten die Proben vermessen werden. Da hierbei erkannt wurde, dass der Farbstoff nach einiger Zeit zerfällt, wurde ein weiterer Praktikumstag eingeplant.\\
Während dem dritten Tag wurde die Kalibrierung wiederholt und eine Sechsfachanalyse, sowie eine Aufstockung mit HMF einer Honigprobe durchgeführt.
 
\chapter{Durchführung}

\label{chap:Durchführung}

Für die Kalibrierung und die Probenvermessung müssen verschiedene Reaktionslösungen angesetzt werden, die einerseits die Matrixeinflüsse verhindern und andererseits für die Farbreaktion benötigt werden. Im folgenden Kapitel sind die verwendeten Chemikalien und Geräte aufgelistet, sowie der genaue Arbeitsablauf gemäß der Vorschrift aus dem Buch Lebensmittelanalytik~\cite{Lebensmittelanalytik} beschrieben.

\section{Verwendete Chemikalien}

Um störende Matrixbestandteile der Honigproben vorab zu entfernen, werden die Carrez-Lösungen I und II benötigt. Die für die Farbreaktion verwendeten Reaktionslösungen können im Chemikalienhandel erworben werden. Dies gilt ebenfalls für die Reinsubstanz HMF, die für die Kalibrierlösungen und die Aufstockung benutzt wird. In der nachfolgenden Tabelle \ref{tab:Chemikalienliste} sind alle benötigten Chemikalien aufgelistet.

\begin{table}[htbp]
	\centering
		\caption{Chemikalienliste}
		\begin{tabular}{L{0.18\linewidth}|C{0.13\linewidth}|C{0.1\linewidth}|c|c|c} 
			Chemikalie & CAS/ Artikel-Nr. & Gefahren-symbol & Reinheit & Hersteller & Lot-Nr.\\
			\hline
			Hydroxymethyl-furfural & 67-47-0 & \includegraphics{../Bilder/Ausrufezeichen.jpg} & 97\% & Alfa Aesar & 10189124\\
			\hline
			p-Toluidin-lösung & 18686.2700 & \includegraphics{../Bilder/Flamme.jpg} \includegraphics{../Bilder/Gesundheitsgefahr.jpg} \includegraphics{../Bilder/Ausrufezeichen.jpg} & 100g/L & Bernd Kraft & 1632697\\
			\hline
			Barbitursäure-lösung & 18685.2700 & - & 5g/L & Bernd Kraft & 1632696\\
			\hline
			Kaliumhexa-cyanoferrat-(II)-Trihydrat & 14459-95-1 & - & $\geq99\%$ & Sigma-Aldrich & SZBC2230V\\
			\hline
			Zinkacetat-Dihydrat & 5970-45-6 & \includegraphics{../Bilder/Ausrufezeichen.jpg} \includegraphics{../Bilder/Umwelt.jpg} & $\geq99,5\%$ & Merck & A0180402 142\\
			\hline
			VE-Wasser & & & & &
		\end{tabular}
	\label{tab:Chemikalienliste}
\end{table}

\begin{figure}[htbp]
	\centering
		\subfigure[p-Toluidinlösung]{
    \includegraphics[]{../Bilder/20150504_140919.jpg}}
		\subfigure[Barbitursäurelösung]{
		\includegraphics[]{../Bilder/20150504_140946.jpg}}
		\subfigure[\small Kaliumhexa-cyanoferrat-(II)-Trihydrat]{
		\includegraphics[]{../Bilder/20150504_141221.jpg}}
		\subfigure[Zinkacetat-Dihydrat]{
		\includegraphics[]{../Bilder/20150504_141229.jpg}}
	\caption{Verwendete Chemikalien}
	\label{fig:Chemikalien}
\end{figure}

Die Sicherheitsdatenblätter sind in Anhang xyz zu finden.

\section{Verwendete Geräte}

Für die Probenvorbereitung und um die verschiedenen Lösungen herzustellen, werden verschiedene Laborgeräte verwendet. Diese sind in der folgenden Tabelle \ref{tab:Geräteliste} zusammengefasst.

\begin{table}[htbp]
	\centering
		\caption{Geräteliste}
		\begin{tabular}{l|C{0.25\linewidth}|c|c|c} 
			Anzahl & Gerät & Volumen in mL & Genauigkeit & Auslaufzeit\\
			\hline
			16 & Messkolben & 10 & A (+/- 0,040mL) & \\
			\hline
			8 & Messkolben & 50 & A (+/- 0,060mL) & \\
			\hline
			4 & Messkolben & 100 & A (+/- 0,100mL) & \\
			\hline
			4 & Vollpipetten & 1 & AS (+/- 0,006mL) & EX\\
			\hline
			2 & Vollpipetten & 2 & AS (+/- 0,010mL) & EX + 15s\\
			\hline
			1 & Vollpipetten & 5 & AS (+/- 0,015mL) & EX + 15s\\
			\hline
			1 & Vollpipetten & 10 & AS (+/- 0,02mL) & EX + 15s\\
			\hline
			1 & Vollpipetten & 20 & AS (+/- 0,03mL) & EX + 15s\\
			\hline
			1 & Messzylinder & 25 & & \\
			\hline
			diverse & Bechergläser & & & \\
			\hline
			1 & Wägeschiffchen & & & \\
			\hline
			1 & Analysenwaage Sartorius M-pact AX224 & max. 120g & d=0,1mg & \includegraphics{../Bilder/20150504_140748.jpg}\\
			\hline
			1 & UV/VIS-Spektralphotometer Varian Cary® 50 & & & \\
			\hline
			1 & Präzisions-Küvette aus opt. Spezialglas & d=10mm & & \\
		\end{tabular}
	\label{tab:Geräteliste}
\end{table}


\section{Proben}

Es sollen acht verschiedenen Honige, ein Zuckerrübensirup und eine Invertzuckermischung vermessen werden. Die folgende Tabelle \ref{tab:Probenliste} zeigt die Probendetails.

\begin{table}[htbp]
	\centering
	\caption{Probenliste}
		\begin{tabular}{C{0.1\linewidth}|C{0.18\linewidth}|c|c|C{0.2\linewidth}|c} 
			Proben-nummer & Probe & Hersteller & Ablaufdatum & Herkunft & Lot-Nr.\\
			\hline
			1 & Flotte Biene Frühlings-blütenhonig & Langnese & 12.2016 & EU-Länder & LM41222\\
			\hline
			2 & Flotte Biene Gebirgs-blütenhonig & Langnese & 09.2016 & Nicht-EG-Länder & LI40946\\
			\hline
			3 & Sommer-blütenhonig & Vom Land & 06.2015 & EG- und Nicht-EG-Länder & L3442714\\
			\hline
			4 & Blütenhonig & Goldland & 03.2015 & EG- und Nicht-EG-Länder & LC40341\\
			\hline
			5 & Mexico & Biophar & 10.2016 & Mexiko & B582725\\
			\hline
			6 & Ägäis & Breitsamer & 07.2016 & Ägäis-Türkei & L4043131\\
			\hline
			7 & Waldhonig & Breitsamer & 10.2016 & Italien, Tschechien & L5644211\\
			\hline
			8 & Zuckerrüben-sirup & Grafschafter & 12.2017 & - & -\\
			\hline
			9 & Winterfutter & Imker B. Hahl & - & Walldorf, Deutschland & -\\
			\hline
			10 & Invertzucker & Pati-Versand.de & 09.2016 & - & 42090.58\\
		\end{tabular}
		\label{tab:Probenliste}
\end{table}

\begin{figure}[htbp]
	\centering
		\includegraphics[width=1.00\textwidth]{../Bilder/20150416_183117.jpg}
	\caption{Übersicht der ersten sechs Honigproben}
	\label{fig:Honigproben}
\end{figure}


\section{Ansetzen der Reaktionslösungen}

Für die Probenaufbereitung werden zwei Carrez-Lösungen benötigt.\\ 
Für die Carrez-Lösung I wird am ersten Praktikumstag 15,1805g Kaliumhexacyanoferrat in einen 100mL Messkolben eingewogen, mit Wasser gelöst und bis zur Ringmarke aufgefüllt.\\ 
Die Carrez-Lösung II wurde zweimal angesetzt, da nach einer Woche Lagerung feste Partikel im Messkolben festgestellt wurden. Am ersten Praktikumstag wurde für die Carrez-Lösung II 30,1485g Zinkacetat in einen 100mL Messkolben eingewogen, mit Wasser im Ultraschallbad gelöst und bis zur Ringmarke aufgefüllt. Da sich das Zinkacetat schlecht auflöste, wurde die Lösung beim zweiten Ansetzen am dritten Praktikumstag leicht erwärmt. Für die zweite Lösung wurde 30,0504g Zinkacetat eingewogen.\\ 
Die p-Toluidinlösung und die Barbitursäurelösung mussten nicht angesetzt werden, da sie in der benötigten Konzentration zur Verfügung standen. In der p-Toluidinlösung waren eine Woche nach Anbruch Feststoffpartikel enthalten.

\section{Ansetzen der Stammlösungen}

Für die Kalibrierreihe werden zwei Stammlösungen mit unterschiedlicher HMF-Konzentration angesetzt. Die Berechnung der benötigten Einwaagen ist in Kapitel \ref{chap:Planung} nachzulesen.\\
\begin{figure}[htbp]
	\centering
		\includegraphics{../Bilder/20150504_140727.jpg}
	\caption{HMF-Reinsubstanz}
	\label{fig:HMF-Reinsubstanz}
\end{figure}
Die HMF-Reinsubstanz wird auf der Analysenwaage in einem Wägeschiffchen eingewogen, mit VE-Wasser in einen 10mL Messkolben überführt und bis zur Ringmarke aufgefüllt.\\
Einwaage SL1: 54,8mg\\
Einwaage SL2: 151,6mg\\
Aus den Einwaagen wird über die im Kapitel \ref{chap:Planung} verwendete Formel zur Konzentrationsberechnung die HMF-Konzentration der beiden Stammlösungen berechnet. Die Konzentration der Stammlösung 1 beträgt 5480mg/L und die Konzentration der Stammlösung 2 beträgt 15160mg/L.\\ 
Von beiden Stammlösungen wird je ein Milliliter in jeweils einen 100mL Messkolben überführt und mit VE-Wasser bis zur Ringmarke aufgefüllt. Somit ergibt sich für die Stammlösung 1.1 eine HMF-Konzentration von 54,8mg/L und für die Stammlösung 2.1 eine HMF-Konzentration von 151,6mg/L. 

\section{Herstellung und Vermessung der Kalibrierlösungen}

Für die Kalibrierlösungen werden aliquote Teile der beiden Stammlösungen 1.1 und 2.1 mit verschiedenen Vollpipetten in 50mL Messkolben abgefüllt und mit einigen Millilitern VE-Wasser vermischt. Die Berechnung der Volumina befindet sich im Kapitel \ref{chap:Planung}. Anschließend werden je 1mL Carrez-Lösung I und II mit einer Vollpipette hinzugefügt. Nach Durchmischen der Lösungen wird mit VE-Wasser bis zur Ringmarke aufgefüllt. Dabei fallen störende Matrixbestandteile als unlösliche Partikel aus.\\
\begin{figure}[htbp]
	\centering
		\includegraphics[width=1.00\textwidth]{../Bilder/20150424_155955.jpg}
	\caption{Kalibrierlösungen mit unlöslichen Partikeln}
	\label{fig:Partikel}
\end{figure}
Die Lösungen werden über einen Faltenfilter filtriert, wobei die ersten 10mL Filtrat verworfen werden. Von dem restlichen Filtrat werden 2mL mit einer Vollpipette entnommen und in einen 10mL Messkolben überführt. In die Kolben werden außerdem jeweils 5mL p-Toluidinlösung und 1mL Barbitursäurelösung pipettiert. Von dem Filtrat der ersten Kalibrierlösung wird ebenfalls die Lösung für den Blindwert angesetzt. Hierbei werden 5mL p-Toluidinlösung zugegeben, aber anstelle der Barbitursäurelösung 1mL VE-Wasser zugesetzt. Zum Homogenisieren werden die Messkolben verschlossen und mehrmals invertiert. Da es sich bei der Farbreaktion um eine Zeitreaktion handelt, werden die Kalibrierlösungen und der Blindwert vor der Vermessung vier Minuten stehen gelassen. Danach müssen die Lösungen zügig vermessen werden, da der Farbkomplex nach dieser Zeit wieder zerfällt.\\
Mit der sechsten Kalibrierlösung wird die Wellenlänge des Absorptionsmaximums zwischen 200 und 800nm bestimmt. Der Ausdruck der Wellenlängenbestimmung ist in Anhang X zu finden.

\[
  \lambda_{max} = 550nm
\]

Dies entspricht der in der Literatur angegebenen Wellenlänge zur Vermessung des Farbkomplexes.~\cite{Winkler}\\
Die Kalibrierlösungen werden bei 550nm gegen den Blindwert vermessen und eine Kalibriergerade erstellt. Um Messfehler auszuschließen wird jede Kalibrierlösung dreimal gemessen.\\
In der nachfolgenden Tabelle \ref{tab:Kalibrierlösungen} sind die Kalibrierlösungen aufgelistet.

\begin{table}[htbp]
	\centering
	\caption{Kalibrierlösungen}
		\begin{tabular}{C{0.1\linewidth}|C{0.1\linewidth}|C{0.15\linewidth}|C{0.15\linewidth}|C{0.15\linewidth}|c} 
			Kalibrier-lösung & Stamm-lösung & Volumen Stamm-lösung\newline in mL & Massenkon-zentration berechnet\newline in mg/L & Massenanteil berechnet\newline in mg/kg & Extinktion\\
			\hline
			1 & 1.1 & 1 & 1,096 & 5,480 & 0,0400\\
			\hline
			2 & 1.2 & 1 & 3,032 & 15,16 & 0,1353\\
			\hline
			3 & 1.1 & 5 & 5,480 & 27,40 & 0,1687\\
			\hline
			4 & 1.1 & 10 & 10,96 & 54,80 & 0,3214\\
			\hline
			5 & 1.2 & 10 & 30,32 & 151,6 & 0,9453\\
			\hline
			6 & 1.2 & 20 & 60,64 & 303,2 & 1,8987\\
		\end{tabular}
		\label{tab:Kalibrierlösungen}
\end{table}

\begin{figure}[htbp]
	\centering
		\includegraphics[width=1.00\textwidth]{../Bilder/20150424_172612.jpg}
	\caption{Kalibrierlösungen}
	\label{fig:Kalibrierlösungen}
\end{figure}


\section{Herstellung und Vermessung der Probelösungen}

Für die Probelösungen werden jeweils ca. 10g Probe in einen 50mL Messkolben eingewogen und in 20mL VE-Wasser gelöst. Die folgende Tabelle \ref{tab:Probeneinwaage} enthält die Lagertemperatur und die Einwaage der einzelnen Proben. 

\begin{table}[htbp]
	\centering
	\caption{Probeneinwaage}
		\begin{tabular}{c|c|c} 
			Probennummer & Lagertemperatur in $^\circ$C & Einwaage in g\\
			\hline
			1 & 25 & 10,2926\\
			\hline
			1 & 60 & 11,0590\\
			\hline
			2 & 25 & 10,2548\\
			\hline
			2 & 60 & 10,0891\\
			\hline
			3 & 25 & 11,3998\\
			\hline
			3 & 60 & 10,2277\\
			\hline
			4 & 25 & 9,9125\\
			\hline
			4 & 60 & 10,0411\\
			\hline
			5 & 25 & 10,0203\\
			\hline
			5 & 60 & 10,2194\\
			\hline
			6 & 25 & 10,0776\\
			\hline
			6 & 60 & 10,0968\\
			\hline
			7 & 25 & 10,4448\\
			\hline
			8 & 25 & 10,4160\\
			\hline
			9 & 25 & 10,1488\\
			\hline
			10 & 25 & 9,9153\\
		\end{tabular}
	\label{tab:Probeneinwaage}
\end{table}

Jeder Probelösung werden je 1mL Carrez-Lösung I und II zugesetzt und die Messkolben nach dem Homogenisieren mit VE-Wasser bis zur Ringmarke aufgefüllt. Die ausgefallenen Partikel werden über Faltenfilter abfiltriert. \\
\begin{figure}[htbp]
	\centering
		\includegraphics[width=1.00\textwidth]{../Bilder/20150427_131648.jpg}
	\caption{Filtration der Proben}
	\label{fig:Filtration}
\end{figure}
Dabei werden die ersten 10mL des Filtrats verworfen. Von dem Filtrat werden mit einer Vollpipette jeweils zweimal 2mL entnommen und in zwei 10mL Messkolben überführt. In einem Messkolben wird der Blindwert angesetzt, im anderen die zu vermessende Probe. In beide Messkolben werden je 5mL p-Toluidinlösung hinzugefügt. Mit einer 1mL Vollpipette wird dem Blindwert 1mL VE-Wasser zugegeben und der Probelösung 1mL Barbitursäurelösung. Beide Messkolben bleiben nach dem Homogenisieren für vier Minuten stehen. Die Messung der Probe erfolgt danach bei 550nm gegen den jeweiligen Blindwert. Die Proben werden jeweils dreimal vermessen um Messfehler auszuschließen.
\begin{figure}[htbp]
	\centering
		\includegraphics[width=1.00\textwidth]{../Bilder/20150427_140221(0).jpg}
	\caption{Probenauswahl mit Blindwerten}
	\label{fig:Probenauswahl}
\end{figure}

\newpage
\section{Mehrfachbestimmung und Aufstockung einer Probe}

Für eine Mehrfachbestimmung wird die Honigprobe 3 am zweiten Praktikumstag einmal und am dritten Praktikumstag sechsmal eingewogen, wie oben beschrieben mit Reaktionslösungen versetzt und anschließend vermessen. Außerdem werden zwei zusätzliche Einwaagen der Honigprobe 3 einmal mit 10mL der Stammlösungen 1.1 und einmal mit 5mL der Stammlösung 2.1 versetzt und so der HMF-Gehalt aufgestockt. Diese beiden Proben werden ebenfalls wie oben beschrieben behandelt und vermessen. Die Einwaagen sind in folgender Tabelle \ref{tab:Probeneinwaage Mehrfachbestimmung + Aufstockungen} aufgeführt:
\begin{table}[htbp]
	\centering
	\caption{Probeneinwaage Mehrfachbestimmung + Aufstockungen}
		\begin{tabular}{c|c|c} 
			Probennummer & Lagertemperatur in $^\circ$C & Einwaage in g\\
			\hline
			3.1 & 25 & 10,323\\
			\hline
			3.2 & 25 & 10,358\\
			\hline
			3.3 & 25 & 10,199\\
			\hline
			3.4 & 25 & 10,835\\
			\hline
			3.5 & 25 & 10,414\\
			\hline
			3.6 & 25 & 10,286\\
			\hline
			3.7 Aufst.1 & 25 & 10,483\\
			\hline
			3.8 Aufst.2 & 25 & 10,365\\
		\end{tabular}
	\label{tab:Probeneinwaage Mehrfachbestimmung + Aufstockungen}
\end{table}

\chapter{Auswertung}
Der HMF-Gehalt der Proben wird mit zwei Methoden bestimmt. Diese werden im folgenden Abschnitt vorgestellt, angewandt und miteinander verglichen. Außerdem wird die Genauigkeit der beiden Methoden über die Wiederfindungsrate überprüft.
\section{Kalibrierung}
Zur Bestimmung von HMF wird eine Kalibrierung durchgeführt. Um die Linearität des Messsignals bei verschiedenen Konzentrationen zu gewährleisten müssen mehrer Kalibrierlösungen vermessen werden. Da ein von 5 bis 300mg/kg großer Bereich abgedeckt werden soll fiel die Entscheidung auf sechs Messpunkte. Die Konzentrationen der Kalibrierstandard sind in folgender Tabelle aufgetragen:

\begin{table}[htbp]
	\centering
		\begin{tabular}{p{0.30\linewidth}|p{0.25\linewidth}|p{0.25\linewidth}|p{0.2\linewidth}}
			Standard & Extinktion & Konzentration / [mg/L] &  Massenanteil / [mg/kg]\\
			\hline
			Std 1 & 0,0400 & 1,004 & 5\\
			\hline
			Std 2 & 0,1353 & 3,032 & 15\\
			\hline
			Std 3 & 0,1687 & 5,020 & 25\\
			\hline
			Std 4 & 0,3214 & 10,040 & 50\\
			\hline
			Std 5 & 0,9453 & 30,320 & 152\\
			\hline
			Std 6 & 1,8987 & 60,640 & 303
		\end{tabular}
	\caption{Kalibrierungen}
	\label{tab:Kalibrierungen}
\end{table}

Die abgegebenen Gehalte und Konzentrationen beziehen sich auf eine theoretische Probeneinwaage von 10,0g Honig.\\
Trägt man nun die Kalibrierpunkte in einem Diagramm ein ergibt sich eine Gerade mit der Funktion y=0,0309*x+0,0175 bei einem Bestimmtheitsmaß von 0,9997.
%Diagramm mit Kalibriergeraden hier einfügen

\section{Quantifizierung mittels Kalibriergerade}
Die durch die Kalibrierung ermittelte Funktion wird zur Bestimmung des Analytgehalts verwendet.
	\[y=m*x+b\]
	\[y=0,0309*x+0,0175\]
	\[x=\frac{ y-0,0175 }{ 0,0309 }\]
Beispielrechnung zu Probe 3 (warm):
	\[x=\frac{ 1,9812-0,0175 }{ 0,0309 }\]
	\[x=63,55mg/L\]
Die Massenkonzentration muss nun noch in den Massenanteil umgerechnet werden.
	\[w[mg/kg]=\frac{ \beta*V }{ 1L * m }*1000000\]
	\[w[mg/kg]=\frac{ 0,06355g/L*0,05L }{ 1L * 10,041g }*1000000\]
	\[w[mg/kg]=316mg/kg\]

	
\section{Quantifizierung mittels Festfaktor}
In der Analysevorschrift zur Bestimmung von HMF in Honig ist folgende Formel angegeben:
	\[HMF[mg/kg]=\frac{ E * 1920 }{ m }\]
Hierbei steht E für die Extinktion, m für die Probenmasse und der Festfaktor 1920 setzt sich zusammen aus dem molaren Extinktionskoeffizienten und der Verdünnung und Umrechnung in die gewünschte Einheit mg/kg.
	\[Faktor=\frac{ M(HMF)*V }{ \epsilon }*1000000\]
	
	\[1920=\frac{ 126g/mol * 0,05L }{ 3,28125L/mol }*1000000\]
Beispielrechnung zu Probe 3 (warm):
	\[HMF[mg/kg]=\frac{ 1,9812 * 1920 }{ 10,041g }\]
	\[HMF[mg/kg]=379mg/kg\]
\section{Ergebnisse der analysierten Proben}
Alle Proben wurden mit den beiden zur Verfügung stehenden Methoden quantifiziert und in einer Tabelle aufgetragen.

\begin{table}[htbp]
	\centering
		\begin{tabular}{p{0.30\linewidth}|p{0.17\linewidth}|p{0.25\linewidth}|p{0.25\linewidth}} 
			Probenbezeichnung & Extinktion & Gehalt HMF über Kalibrierung \ [mg/kg] &  Gehalt HMF über Festfaktor \ [mg/kg]\\
			\hline
			1 Flotte Biene Frühlingsblütenhonig (kalt) & 0,016 & <5 & <5\\
			\hline
			1 Flotte Biene Frühlingsblütenhonig (warm) & 1,6041 & 256 & 307\\
			\hline
			2 Flotte Biene Gebirgsblütenhonig (kalt) & 0,0102 & <5 & <5\\
			\hline
			2 Flotte Biene Gebirgsblütenhonig (warm) & 1,3184 & 210 & 252\\
			\hline
			3 Sommerblütenhonig (kalt) & 0,1500 & 20 & 27\\
			\hline
			3 Sommerblütenhonig (warm) & 1,9812 & 316 & 379\\
			\hline
			4 Blütenhonig (kalt) & 0,0621 & 7 & 12\\
			\hline
			4 Blütenhonig (warm) & 1,5500 & 245 & 294\\
			\hline
			5 Mexico (kalt) & 0,0639 & 8 & 12\\
			\hline
			5 Mexico (warm) & 1,2237 & 194 & 234\\
			\hline
			6 Ägäis (kalt) & 0,0871 & 11 & 16\\
			\hline
			6 Ägäis (warm) & 1,1639 & 185 & 223\\
			\hline
			7 Waldhonig & 0,0362 & <5 & 7\\
			\hline
			8 Zuckerrübensirup & n.B. & n.B. & n.B.\\
			\hline
			9 Winterfutter & 0,0255 & <5 & 5\\
			\hline
			10 Invertzucker & 1,541 & 249 & 298\\

		\end{tabular}
	\caption{Messergebnisse}
	\label{tab:Messergebnisse}
\end{table}

\section{Wiederfindungsrate}
Da zur Bestimmung des HMF-Gehalts nur ideale Kalibrierlösungen bzw. der angegebene Festfaktor verwendet wurden, muss der Einfluss der Probenmatrix auf das Analysenergebnis ermittelt werden. Hierzu wird die Wiederfindungsrate anhand einer Aufstockung ermittelt. Die Wiederfindungsrate (WFR) ist der Quotient aus dem Istwert und dem Sollwert der Probe. Zur Berechnung der Wiederfindung wird die folgende Formel verwendet:
	\[WFR=\frac{ Gefundener Gehalt + Istwert }{ Gefundener Gehalt + Sollwert } *100 \]
Als Gefundener Gehalt wird der Mittelwert aus der Sechsfachbestimmung der Probe 3 verwendet.
	\[x(Mittelwert)=\frac{ x1+x2...xn }{ n } \]
Die Berechnung des Mittelwertes nach dem in der Vorschrift angegeben  Festfaktor:
	\[28mg/kg=\frac{ 29mg/kg+31mg/kg+25mg/kg+25mg/kg+31mg/kg+28mg/kg }{ 6 } \]
Die Berechnung des Mittelwertes nach der erstellten Kalibriergerade:
	\[21mg/kg=\frac{ 22mg/kg+23mg/kg+18mg/kg+19mg/kg+23mg/kg+21mg/kg }{ 6 } \]
Nach dem gemittelten Gehalt muss der aufgestockte Anteil an Analyt berechnet werden. In eine Aufstockung wurden 10ml der Stammlösung 1.1 zugegeben. Das entspricht einer Masse von 0,502 mg. Bezogen auf die Probeneinwaage von 10,483g ergibt sich ein theoretischer Massenanteil von:  
	\[w=\frac{ m(Analyt) }{ m(Gesamt) } \]
	\[47,9mg/kg=\frac{ 0,502mg }{ 0,010483kg } \]
In einer zweiten Aufstockung wurden 5ml Stammlösung 2.1 zugegeben, was einer Masse von 0,758mg entspricht. Auch hier wird mit der Einwaage von 10,365g der theoretische Massenanteil berechnet:
	\[73,1mg/kg=\frac{ 0,758mg }{ 0,010365kg } \]
Mit den Mittelwerten und den theoretischen kann die Wiederfindungsrate berechnet werden.\\
Zuerst die erste Aufstockung anhand des Festfaktors:
	\[92,3=\frac{ 28mg/kg + 47,9mg/kg }{ 82mg/kg } *100 \]
Und nach der Kalibriergeraden:
	\[104,4=\frac{ 21mg/kg + 47,9mg/kg }{ 66mg/kg } *100 \]
Anhand der beiden Wiederfindungsraten kann man darauf schließen, dass die Quantifizierung mit der Kalibriergeraden richtigere Werte liefert. Zur Kontrolle werden beide Berechnungen noch einmal mit der zweiten Aufstockung durchgeführt.\\
Bestimmung der zweiten Aufstockung nach Festfaktor:
	\[95,4=\frac{ 28mg/kg + 73,1mg/kg }{ 106mg/kg } *100 \]
Bestimmung der zweiten Aufstockung nach Kalibriergeraden:
	\[108,2=\frac{ 21mg/kg + 73,1mg/kg }{ 87mg/kg } *100 \]
Beide Wiederfindungsraten sind nahe 100 \% jedoch findet man mit der Kalibriergeraden tendenziell leicht erhöhte Gehalte an HMF wieder, während man in den Proben weniger HMF findet. Es ist auch bemerkenswert wie gut der angegebene Festfaktor zur Gehaltsbestimmung geeignet ist obwohl bei seiner Verwendung in keinster Weise kalibriert wird und das verwendete Messgerät damit völlig ignoriert wird. Auch bei der Richtigkeit der Ergebnisse sind beide Methoden vergleichbar.
	
\section{Standardaddition}
Da die Probe 3 zweimal mit unterschiedlichen Mengen an HMF aufgestockt wurde bietet sich ein Quantifizierung über Standardaddition an. So lässt sich auch der Einfluss der, nach der Abtrennung noch enthaltenen, Probenmatrix eliminieren. Eine Übereinstimmung mit unseren anderen Ergebnissen würde für einen nur Unwesentlichen Einfluss der verbliebenen Matrixanteile sprechen. Der Massenanteil mit mehreren Aufstockungen wird berechnet in dem man die Extinktion der Probe(hier die Mittelwerte der sechs Bestimmungen von Probe 3) sowie die Extinktionen der Aufstockungen zur Erstellung eines Diagramms verwendet. Die Probe ohne Aufstockung markiert den Nullpunkt der X-Achse, jede Aufstockung ist auf der X-Achse gemäß ihrer zugesetzten Konzentration aufgetragen. 
%Diagramm hier einfügen
Zieht man durch die erhaltenen Punkte eine Gerade und verlängert sie zurück bis sie die X-Achse im negativen Bereich schneidet, kann man an diesem Punkt die Konzentration der Probe graphisch ablesen. Der Gehalt der Probe lässt sich zudem über die lineare Regression der Geraden bestimmen. Sie lautet bei dieser Analyse:
	\[y=0,0058*x+0,1572\] 
Setzt man y nun gleich 0 und löst nach x auf erhält man die Konzentration der Probe in mg/kg.
 	\[0=0,0058*x+0,1572\] 
	\[x=\frac{ -1,572 }{ 0,0058 }   |*-1\] 
	\[x=27mg/kg\] 
Der über Standardaddition ermittelte Gehalt an HMF in der Probe 3 ist damit 27mg/kg und entspricht damit praktisch dem Gehalt der mit den anderen Quantifizierungsmethoden ermittelt wurde.
\chapter{Fazit}
Der HMF-Gehalt aller vermessenen unbehandelten Honige waren bezüglich der staatlichen Vorgaben unbedenklich. Selbst der Honig, der das Mindesthaltbarkeitsdatum bereits überschritten hatte, blieb innerhalb der Grenzwerte. Allerdings war ein Großteil bereits auskristallisiert. Somit war die vermessene Probe inhomogen. Es kann nicht ausgeschlossen werden, dass ein erhöhter HMF-Gehalt in den kristallisierten Rückständen vorliegt.\\
Der Invertzucker überraschte mit seinem hohen HMF-Gehalt von ca. 300mg/kg. Dies erklärt sich durch sein Herstellungsverfahren bei dem eine Saccharoselösung mit Zitronensäure gekocht wird.\\
Der Sommerblütenhonig der Firma ''Vom Land'' (Probe 3 \ref{tab:Messergebnisse}) zeigt den höchsten HMF-Gehalt mit ca. 30mg/kg. Die Probe war nahe dem Verfallsdatum, hatte dieses aber noch nicht überschritten.\\
Der thermische Einfluss auf die Honige führte zu einem deutlichen Anstieg des HMF-Gehalts. Die Konzentrationen bewegten sich bei allen Honigen bei ca. 250mg/kg. Da alle Honige für die gleiche Zeitspanne temperiert wurden, kann man annehmen, dass die Entstehungsrate von HMF bei allen Proben annähernd identisch ist. Eine Ausnahme bildet der Sommerblütenhonig, bei dem der HMF-Gehalt mit 379mg/kg außerhalb des linearen Messbereichs liegt. Somit kann keine exakte Aussage über die HMF Bildung in der Probe 3 getroffen werden. Allerdings korelliert dieser hohe Wert mit dem gefunden HMF-Gehalt in der unbehandelten Probe 3.\\
Das Winterfutter wies einen äußerst niedrigen HMF-Gehalt auf. Dies zeigt, dass der Imker in der Winterphase den Bienenstock nicht zusätzlich beheizt hat. Das Beheizen erniedrigt den Energieaufwand, den die Bienen im Winter erbringen müssen um ihren Stock warm zu halten. Höhere Temperaturen im Stock führen zu einem frühen Brüten der Bienen.\\
Vergleicht man die über die Kalibrierung ermittelten Werte mit den über den Festfaktor berechneten, so zeigt sich das die berechneten Werte höher sind. Die Ursache hierfür ist wahrscheinlich ein systematischer Fehler, da die Kalibriergerade mit einem Bestimmtheitsmaß von 0,9997 in sich schlüssig ist. Der Grund dafür könnte sein, dass nicht alle Matrixbestandteile bei der Fällung und anschließender Filtration abgetrennt werden konnten.\\
Eine verwendbare Kalibriergerade konnte erst im zweiten Versuch erstellt werden. Beim ersten Vermessen der Kalibrierlösungen wurde nicht die enorme Zerfallsrate des Farbkomplexes berücksichtigt. Nach WINKLER besteht nur ein Zeitfenster von einer Minute um die Messung durchzuführen.~\cite{Winkler} Die nicht verwendete Kalibriergerade ist im Anhang hinzugefügt.\\
Der Rübensirup der Firma ''Grafschafter'' (Probe 8) war nicht zur Vermessung geeignet. Zum einen war die Eigenfärbung dunkelbraun bis tiefschwarz. Eine Lichtdurchlässigkeit war unwahrscheinlich. Bei der Filtration setzten sich die Poren des Filters sofort zu. Es konnte kein Filtrat gewonnen werden.\\
%Bild Grafschafter
Die durchgeführten Versuche zeigen einen ersten Einblick in das Zusammenwirken von Fruchtzucker und HMF bei thermischer Belastung sowie bei Raumtemperatur. Zahlreiche Einflussfaktoren blieben bei den Versuchen jedoch bislang unberücksichtigt. So liegen keine Erkenntnisse über den Zusammenhang von pH-Wert, Wassergehalt und Enzymtätigkeit mit der Bildung von HMF vor. Außerdem ist nicht bekannt ob der Gesamte Fruchtzucker in Honig zu HMF umgewandelt werden kann, oder ob sich bei hoher Konzentration ein Gleichgewicht einstellt. hierfür müssten weitere Untersuchungen, auch über einen längeren Zeitraum, durchgeführt werden.
%tropischer Honig nicht auffällig
\chapter{Vorschrift zur photometrischen Bestimmung von Hydroxymethylfurfural}

Hydroxymethylfurfural (HMF) entsteht bei der thermischen Zersetzung von Fructose und kann z.B. in Honig vorkommen.\\
Die Probe wird mit p-Toluidin und Barbitursäure versetzt und die gefärbte Lösung photometrisch vermessen.

\section{Chemikalien}
- Hydroxymethylfurfural\\
- p-Toluidinlösung, 100g/L in 2-Propanol\\
- Barbitursäurelösung, 5g/L\\
- Kaliumhexacyanoferrat(II)-Trihydrat\\
- Zinkacetat-Dihydrat\\
- VE-Wasser

\section{Bestimmung}
\subsection{Herstellung der Carrez-Lösung I}
15,0g Kaliumhexacyanoferrat(II)-Trihydrat werden in einem 100mL Messkolben in VE-Wasser gelöst und bis zur Ringmarke aufgefüllt.
\subsection{Herstellung der Carrez-Lösung II}
30,0g Zinkacetat-Dihydrat werden in einem 100mL Messkolben in VE-Wasser gelöst und bis zur Ringmarke aufgefüllt.
\subsection{Herstellung der Kalibrierlösungen}
Mit der Kalibrierung soll der Bereich von 5 bis 300ppm HMF abgedeckt werden. Die Kalibrierlösungen sind entsprechend der Probelösung anzusetzen.
\subsection{Herstellung der Probelösung}
10g Honig wird in einen 50mL Messkolben eingewogen und ohne Erwärmen in 20mL VE-Wasser gelöst. Nacheinander werden je 1mL Carrez I und II zugegeben und nach Durchmischen mit VE-Wasser bis zur Ringmarke aufgefüllt. Die Lösung wird durch einen trockenen Faltenfilter filtriert, wobei die ersten 10mL verworfen werden.
\subsection{Photometrische Messung}
In zwei 10mL Messkolben werden je 2mL Probenlösung pipettiert und je 5mL p-Toluidinlösung (Achtung: giftig, PSA benutzen und im Abzug arbeiten) zugegeben. In den einen Messkolben wird 1mL VE-Wasser pipettiert und in den anderen Messkolben 1mL Barbitursäurelösung. Danach werden beide Mischungen durchmischt.\\ 
Reaktionszeit: 3 - 4 Minuten\\
Achtung der Farbstoff zerfällt nach Ablauf dieser Zeit wieder.\\
Die Messwellenlänge ist mittels eines Lambdascans im sichtbaren Bereich zu bestimmen.
\subsection{Auswertung}
Der Gehalt an HMF in mg/kg berechnet sich wie folgt:\\
\[HMF[mg/kg]=\frac{ E * 1920 }{ m }\]
Hierbei steht E für die Extinktion, m für die Probenmasse und der Festfaktor 1920 setzt sich zusammen aus dem molaren Extinktionskoeffizienten, der Verdünnung und Umrechnung in die gewünschte Einheit mg/kg.\\
Des Weiteren wird eine Kalibriergerade erstellt und die Proben über diese ausgewertet.
Die Ergebnisse sind zu vergleichen und Rückschlüsse zu ziehen.\\
Diese Vorschrift basiert auf der Anleitung zur ``Photometrischen Bestimmung von Hydroxymethylfurfural'' aus dem Buch ``Lebensmittelanalytik \cite{Lebensmittelanalytik}''.

\listoffigures
\listoftables

\bibliographystyle{gerunsrt}
\bibliography{Quellen}

\end{document}
